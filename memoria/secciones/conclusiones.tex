% CONCLUSIONES

\chapter{Conclusiones}
\label{conclusiones}


Este trabajo ha explorado en profundidad la aplicación de sistemas de generación aumentada por recuperación (RAG) basados en grandes modelos del lenguaje (LLMs) en el contexto de la Constitución Española. Se ha logrado desarrollar un sistema capaz de responder preguntas relacionadas con dicho documento, integrando metodologías avanzadas de chunkerización y técnicas de recuperación de información.

Se compararon diversas estrategias de chunkerización, evaluando su eficiencia y relevancia en la recuperación de fragmentos textuales adecuados para responder preguntas sobre la Constitución Española. Además, se implementó un agente basado en LLMs que emplea técnicas de recuperación y generación para responder consultas de manera precisa y contextualizada.

En resumen, las principales contribuciones de este trabajo son:

\begin{enumerate}
\item El desarrollo de un splitter personalizado adaptado para fragmentar artículos de la Constitución Española, optimizando la recuperación de información relevante para su uso en sistemas RAG.
\item La comparación y evaluación detallada de diferentes técnicas de chunkerización aplicadas a textos legales, proporcionando una guía sobre las estrategias más eficaces para este tipo de documentos.
\item La implementación de un agente basado en LLMs que utiliza una arquitectura RAG, integrando recuperación de información y generación de texto para ofrecer respuestas precisas y contextualizadas sobre la Constitución Española.
\end{enumerate}

Aunque existen limitaciones, las perspectivas futuras para la evolución de estos sistemas son prometedoras, especialmente a medida que se avanza en el desarrollo de modelos más potentes y se mejora la capacidad de recuperación de información relevante. Con mejoras continuas, este tipo de herramientas podrían convertirse en recursos fundamentales tanto en el ámbito jurídico como en otras disciplinas que requieran el manejo de grandes volúmenes de información técnica y precisa.